\chapter{Machine Learning trong phân tích dữ liệu bảng (Panel Data)}
\section{Xử lý dữ liệu bảng lớn với ML}
\subsection{Giới thiệu}
Dữ liệu bảng (panel data) là loại dữ liệu kết hợp giữa chuỗi thời gian và dữ liệu chéo. Xử lý dữ liệu bảng lớn với Machine Learning (ML) đòi hỏi các phương pháp tối ưu hóa, mô hình phi tuyến và các thuật toán có khả năng khai thác mối quan hệ phức tạp trong dữ liệu.

\subsection{Mô hình hóa dữ liệu bảng}
Một mô hình dữ liệu bảng tổng quát có dạng:
\begin{equation}
    y_{it} = \alpha + X_{it} \beta + u_{it},
\end{equation}
với:
\begin{itemize}
    \item $y_{it}$ là biến phụ thuộc của cá thể $i$ tại thời điểm $t$.
    \item $X_{it}$ là vector các biến độc lập.
    \item $\beta$ là vector các hệ số hồi quy.
    \item $u_{it}$ là sai số.
\end{itemize}

Trong Machine Learning, ta có thể sử dụng các mô hình phi tuyến như Random Forest, Gradient Boosting hoặc Neural Networks để ước lượng hàm hồi quy:
\begin{equation}
    y_{it} = f(X_{it}) + \epsilon_{it},
\end{equation}
nơi $f(X_{it})$ là một hàm phi tuyến được học bởi mô hình ML.

\subsection{Các phương pháp Machine Learning trong dữ liệu bảng}
\subsubsection{Random Forest và Gradient Boosting}
Random Forest và Gradient Boosting Trees (GBT) là các thuật toán ML phổ biến để xử lý dữ liệu bảng lớn. Chúng có thể bắt được các quan hệ phi tuyến và tương tác giữa các biến.

\subsubsection{Mô hình mạng nơ-ron nhân tạo (Neural Networks)}
Mạng nơ-ron có thể mở rộng để xử lý dữ liệu bảng bằng cách sử dụng kiến trúc như Long Short-Term Memory (LSTM) hoặc Transformer để nắm bắt quan hệ theo thời gian giữa các quan sát.

\subsection{Ước lượng tham số và đánh giá mô hình}
Các phương pháp ước lượng tham số phổ biến bao gồm Regularized Regression (Ridge, Lasso), Bayesian Inference và Maximum Likelihood Estimation (MLE). Các tiêu chí đánh giá mô hình gồm:
\begin{itemize}
    \item Mean Squared Error (MSE)
    \item Mean Absolute Error (MAE)
    \item R-squared ($R^2$)
    \item Bayesian Information Criterion (BIC), Akaike Information Criterion (AIC)
\end{itemize}

\subsection{Kết luận}
Machine Learning mở ra nhiều hướng mới trong phân tích dữ liệu bảng lớn, giúp nâng cao khả năng dự báo và phân tích chính sách kinh tế lượng.




\section{So sánh Fixed Effects, Random Effects với ML}
\subsection{Giới thiệu}
Trong kinh tế lượng dữ liệu bảng (panel data), hai phương pháp phổ biến để kiểm soát các hiệu ứng không quan sát được là mô hình hiệu ứng cố định (Fixed Effects - FE) và mô hình hiệu ứng ngẫu nhiên (Random Effects - RE). Cùng với sự phát triển của Machine Learning (ML), các phương pháp học máy cũng được áp dụng để xử lý dữ liệu bảng.

\subsection{Mô hình Hiệu ứng Cố định (Fixed Effects Model)}
Mô hình hiệu ứng cố định có dạng tổng quát:
\begin{equation}
    y_{it} = \beta X_{it} + \alpha_i + \varepsilon_{it},
\end{equation}
trong đó:
\begin{itemize}
    \item $y_{it}$ là biến phụ thuộc của cá nhân $i$ tại thời điểm $t$,
    \item $X_{it}$ là vector các biến độc lập,
    \item $\alpha_i$ là hiệu ứng cố định của cá nhân $i$,
    \item $\varepsilon_{it}$ là sai số.
\end{itemize}
Mô hình FE kiểm soát sự không đồng nhất giữa các cá nhân bằng cách đưa vào các hiệu ứng cố định $\alpha_i$, ước lượng thường được thực hiện bằng phương pháp bình phương bé nhất trong nhóm (Within estimator).

\subsection{Mô hình Hiệu ứng Ngẫu nhiên (Random Effects Model)}
Mô hình hiệu ứng ngẫu nhiên có dạng:
\begin{equation}
    y_{it} = \beta X_{it} + u_i + \varepsilon_{it},
\end{equation}
trong đó $u_i \sim N(0, \sigma_u^2)$ là thành phần ngẫu nhiên đại diện cho hiệu ứng cá nhân. Mô hình RE giả định rằng $u_i$ không tương quan với $X_{it}$, do đó có thể ước lượng bằng bình phương bé nhất tổng quát (GLS).

\subsection{Ứng dụng Machine Learning trong Dữ liệu Bảng}
Machine Learning (ML) có thể thay thế hoặc kết hợp với các mô hình FE/RE để dự báo và phân tích dữ liệu bảng. Một số phương pháp ML phổ biến:
\begin{itemize}
    \item Cây quyết định (Decision Trees), Rừng ngẫu nhiên (Random Forest), XGBoost: Có thể phát hiện mối quan hệ phi tuyến giữa các biến.
    \item Hồi quy Ridge, Lasso: Kiểm soát đa cộng tuyến tốt hơn FE/RE.
    \item Mạng nơ-ron nhân tạo (Neural Networks): Có thể xử lý dữ liệu bảng phức tạp.
\end{itemize}

Một cách tiếp cận kết hợp là sử dụng Fixed Effects trong bước tiền xử lý để loại bỏ các yếu tố không quan sát được, sau đó áp dụng ML để cải thiện dự báo.

\subsection{Kết luận}
Mô hình hiệu ứng cố định và hiệu ứng ngẫu nhiên có lợi thế khi dữ liệu có cấu trúc bảng truyền thống, nhưng ML mang lại khả năng phát hiện mô hình phức tạp hơn. Việc kết hợp cả hai phương pháp có thể giúp nâng cao độ chính xác trong dự báo kinh tế lượng.





\section{Ứng dụng ML vào phân tích tác động theo thời gian}
\subsection{Giới thiệu}
Phân tích tác động theo thời gian là một phương pháp quan trọng trong kinh tế lượng và khoa học dữ liệu nhằm xác định sự thay đổi của một biến phụ thuộc theo thời gian do tác động của một yếu tố nhất định. Machine Learning (ML) có thể hỗ trợ trong việc phân tích dữ liệu thời gian thực, cải thiện độ chính xác và phát hiện các mẫu phức tạp.

\subsection{Mô hình hóa tác động theo thời gian}
Giả sử chúng ta có một tập dữ liệu chuỗi thời gian $\{y_t, X_t\}_{t=1}^{T}$, trong đó $y_t$ là biến phụ thuộc và $X_t$ là tập hợp các biến độc lập.

\subsubsection{Mô hình truyền thống}
Một cách tiếp cận phổ biến trong kinh tế lượng là sử dụng mô hình hồi quy:
\begin{equation}
    y_t = \beta_0 + \beta_1 X_t + \epsilon_t,
\end{equation}
trong đó $\epsilon_t \sim \mathcal{N}(0, \sigma^2)$ là nhiễu trắng.

Tuy nhiên, mô hình này không thể nắm bắt được các yếu tố phi tuyến hoặc sự phụ thuộc phức tạp theo thời gian.

\subsubsection{Ứng dụng Machine Learning}
Machine Learning có thể cải thiện phân tích bằng cách sử dụng các mô hình như:

\begin{itemize}
    \item \textbf{Random Forest}: Một tập hợp các cây quyết định có thể nắm bắt mối quan hệ phi tuyến.
    \item \textbf{XGBoost}: Một phương pháp boosting giúp tối ưu hóa dự báo chuỗi thời gian.
    \item \textbf{Recurrent Neural Networks (RNNs)}: Mạng nơ-ron hồi quy giúp phát hiện xu hướng và sự phụ thuộc theo thời gian.
\end{itemize}

Ví dụ, sử dụng Random Forest để dự báo tác động theo thời gian:
\begin{equation}
    y_t = f(X_t) + \epsilon_t,
\end{equation}
trong đó $f(\cdot)$ là một hàm phi tuyến ước lượng bởi mô hình ML.

\subsection{Ước lượng và đánh giá mô hình}
Để đánh giá mô hình, ta có thể sử dụng các chỉ số:
\begin{itemize}
    \item \textbf{MSE (Mean Squared Error)}:
    \begin{equation}
        \text{MSE} = \frac{1}{T} \sum_{t=1}^{T} (y_t - \hat{y}_t)^2.
    \end{equation}
    \item \textbf{MAPE (Mean Absolute Percentage Error)}:
    \begin{equation}
        \text{MAPE} = \frac{100\%}{T} \sum_{t=1}^{T} \left| \frac{y_t - \hat{y}_t}{y_t} \right|.
    \end{equation}
    \item \textbf{R-squared}:
    \begin{equation}
        R^2 = 1 - \frac{\sum_{t=1}^{T} (y_t - \hat{y}_t)^2}{\sum_{t=1}^{T} (y_t - \bar{y})^2}.
    \end{equation}
\end{itemize}

\subsection{Kết luận}
Ứng dụng Machine Learning vào phân tích tác động theo thời gian giúp phát hiện các mối quan hệ phức tạp, cải thiện độ chính xác và dự báo tốt hơn so với các phương pháp truyền thống. Tuy nhiên, việc lựa chọn mô hình phù hợp và kiểm định tính ổn định của mô hình vẫn là một thách thức quan trọng.
